\chapter{Over-current protection} \label{sec:OC}

% Follow the same format as the previous chapters - circuit diagram, and then simulated and measured results. 

\section{Design} \label{sec:sw_design}

Over current protection is protection against excessive currents or current beyond the acceptable current rating of equipment. It generally operates instantly \cite{OverC}. In this design an SR latch \cite{SR_Latch} was implemented to output the trip status of the current switch. The logic of an SR Latch can be seen in table \ref{SRTT}. The switch is in a conducting state when S=1 and R=0 and the switch is in a non-conducting  when S=0 and R=1. S is represented as "$S_{beetle}$" in the circuit diagram in Fig. \ref{fig:Latch_OC}. This input remains a digital low (0 volt) as controlled by one of the digital pins of the Arduino Leonardo \cite{beetle} to remain in a latch state (S=0 and R=0) till the latch output is no longer conducting. Then a pulse is sent via S to reset the trip again. R is represented as "$R_{comparing \ 200mA}$" in Fig. \ref{fig:Latch_OC}. R is the output of a comparator. The comparator constantly compares the output of the current transducer \ref{sec:itrans} to a reference voltage level representing 200 mA. When the current transducer output is below this reference level, then R=0. When the current transducer output is above this reference level, then R=1.



\begin{table}[H]
\centering
\caption{SR Latch truth table}
\begin{tabular}{|c|c|c|}
\hline
S & R & Q     \\ \hline
0 & 0 & Latch \\ \hline
0 & 1 & 0     \\ \hline
1 & 0 & 1     \\ \hline
1 & 1 & Unknown     \\ \hline
\end{tabular}
\label{SRTT}
\end{table}



The design of the SR latch was directly obtained from \cite{SR_Latch}, but the transistors were swapped for 2N3904 BJTs. The two latch BJTs in Q1 and Q2 in \ref{fig:Latch_OC} act as switches.

\begin{itemize}
    \item $R_8$ and $R_9$ act as high impedance input resistors to limit the effect of current from $R_{comparing \ 200mA}$ and $S_{beetle}$ voltage sources such that do not affect the logic operation of the latch. Let $R_8=R_9=\SI{10}{\kilo \ohm}$.
    \item $R_6$ and $R_7$ are used to limit base currents, let $R_6=R_7=\SI{100}{\kilo \ohm}$.
    \item $R_{10}$ and $R_{11}$ are used as current-limiting resistors. If we design for a "On" state for the transistors, we design for a collector current of \SI{5}{\milli \ampere}, then $R_{10}=R_{11}=\frac{5}{1m}=\SI{1}{\kilo \ohm}$.
    \item $R_{12}$ and $R_{13}$ also act as current-limiting resistors to limit interaction between the bases and collectors of Q1 and Q2.
    Set $R_{12}=R_{13}=\SI{10}{\kilo \ohm}$.
\end{itemize}

\begin{figure}[H]
    \centering
    \includegraphics[trim={5cm 0 5cm 0},clip,scale=0.7]{A3/Latch1.pdf}
    \caption{Over-current operational diagram including SR latch, current-supported voltage buffer, comparator and optocoupling-triac switch}
    \label{fig:Latch_OC}
\end{figure}

% trim={<left> <lower> <right> <upper>}

With regards to the current-supported voltage buffer represented by U5 and Q3 with $R_{14}$ in Fig. \ref{fig:Latch_OC}, it is used due to problems experienced in simulation where the output of the SR latch would present a logical high (5 volt), but when directly connected to U6 (the optocoupling-triac switch), the output would present as 1 volt (not enough voltage to turn on the internal LED of the opto-coupling triac). A normal voltage follower was then used, but the problem persisted. A  current-supported voltage buffer was then added to resolve the problem. The resistor $R_{14}$ can be increased till the output voltage of the buffer is more than the typical forward voltage, $V_{\gamma}=\SI{1.2}{\volt}$ of the MOC3020 \cite{MOC_opto}. The current will then also be enough to fully turn on the internal LED, thereby letting the source voltage drop over the load. A \hl{TVS} is connected over the load to protect from over-voltage transients. We were given he choice to use a P6KE43CA TVS \cite{P6KE} or BZW06-28B TVS \cite{BZW06}. The BZW06-28B TVS was chosen due to a higher steady state power dissipation on infinite heat sink of 5 W compared to 1.7 of the BZW06-28B TVS.

\section{Simulation} \label{sec:sw_simu} 

\begin{figure}[H]
    \centering
    \includegraphics[scale=0.35]{A3/LatchRes1.pdf}
    \caption{Simulation of trip-point with ramping function at comparator input}
    \label{fig:Latch_OC_Res}
\end{figure}

Fig. \ref{fig:Latch_OC_Res} shows a ramping function represented as "V(ramp)" that simulates the current transducer input to the comparator. It makes the comparator high when exceeding the reference voltage level at 3 volt of U7. In turn "V(Q\_trip)" is made low and the opto-coupling triac is in a non-conducting state. Before the trip point, the switch is in a conducting state represented by "V(V\_mains\_TRIAC\_result)".

\section{Measurements} \label{sec:sw_meas}

\begin{figure}[H]
\centering
\begin{subfigure}{.5\textwidth}
     \centering
    \includegraphics[scale=0.8]{A3/tripoverall.JPG}
    \caption{Trip measurement showing the switch stays off and also illustrating dead-time of the source voltage}
    \label{subfig:trip1}
\end{subfigure}%
\begin{subfigure}{.5\textwidth}
     \centering
    \includegraphics[scale=0.8]{A3/measutrip.JPG}
    \caption{Accurate measurement of the trip time}
    \label{subfig:trip_accurate}
\end{subfigure}
\label{trip measurements}
\caption{Switch trip time measurements}
\end{figure}

Fig. \ref{subfig:trip1} shows that when an opto-coupling triac is used, the load is not continuous anymore because dead time is apparent. This is due to the time that the triac takes to turn on every cycle.
Fig. \ref{subfig:trip_accurate} shows an accurate measure on the time that the switch takes to turn on due to trip. It is measured as approximately \SI{7.5}{\milli s}. We compare this to the simulated trip time which is approximately \SI{110}{\micro s} measured from Fig. \ref{fig:sim_tripttt}. 

\begin{figure}[H]
    \centering
    \includegraphics[scale=0.35]{A3/tripttt1.pdf}
    \caption{Simulated trip time}
    \label{fig:sim_tripttt}
\end{figure}

Although our practical system measurements is relatively far from our simulated value, the trip time is still less than \SI{150}{\milli s} as described in \cite{assignment_3}. Thus, our design is acceptable since the load voltage is completely switched off without any back-EMF voltage and stays completely off.




\section{Summary and implementation}
% State whether your design performs as expected. 
We can conclude that the over-voltage and over-current design worked efficiently enough such that the circuit trip every time it was expected to. Implementation was very simple and results were accurate enough to meet project specifications.

\begin{figure}[H]
    \centering
  	\includegraphics[scale=0.4]{Picures/overcurrentpic.PNG}
	\caption{Implementation of the over-current circuitry} \label{subfig:OC_pcb}
 \end{figure}
% trim={<left> <lower> <right> <upper>}





























