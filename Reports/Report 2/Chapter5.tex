\chapter{System tests}

\begin{figure}[H]
    \centering
    \includegraphics[scale=.4]{Figures/FullPCB.JPG}
    \caption{Final constructed circuit with student card and bar-coded ID}
\end{figure}


\begin{table}[H]
\centering
\begin{tabular}{|c|c|c|}
\hline
Signal & Noise Maximum {[}mV{]} & Noise minimum {[}mV{]} \\ \hline
+5 V   & 36.0            &-38.0         \\ \hline
-5 V   & 64.0                       & -32.0                       \\ \hline
\end{tabular}
\caption{Table of supply voltage noises}
\label{supply_noise_table}
\end{table}
 The above table shows noise values when all transducers are simultaneously working and is obtained from figures \ref{fig:+5V_noise} and  \ref{fig:-5V_noise} in Appendix D.
A 100 $\Omega$ load was used measure current. The voltage before and after the resistor can be seen in Fig. \ref{fig:system_current} in Appendix D. Using KCL for full system load: $$I_{system}=\frac{V_{max_{peak}}-V_{min_{peak}}}{100}=\frac{29.2-38.4m}{100}\approx\SI{291}{\milli\ampere}$$ 
% Here you include an image of your PCB (showing your barcoded ID) and any system-level tests you deem appropriate. For example, the total current drawn by the opamp circuitry. 










