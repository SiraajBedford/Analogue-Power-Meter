\chapter{Current peak transducer}

\section{Theory and related work} \label{sec:current_theory}

A current peak transducer is configured to follow the peak voltage over a current sense resistor. The peak voltage is monitored when different loads that (in our case a purely resistive load for testing) induce different current levels are to be monitored. A positive peak detector captures the most positive peak of the voltage signal at its input \cite{precision_peak_detector}. Inaccurate peak detection circuits only require a diode and capacitor combination to charge up to the approximate peak voltage, but since we want to accurately follow the peak-to-peak voltage over a small current sense resistor which is amplified using a non-inverting amplifier \cite{non-inverting}, a precision peak detector, analogous to the one used in Section \ref{sec:design_vpd1} can be used to get rid of this error.

\section{Design} \label{sec:design_current}

The $0 \ to \ 285$ mA will be converted into a DC analogue voltage in the following manner:
\begin{itemize}
    \item Different load currents will be emulated by using different resistive load sizes. Consider the case of he maximum load current of $285$ mA as an example. This  will be when $R_{load}=100 \ \Omega$.
    \item This will result in the voltage over $R_{sense}$ being a maximum, i.e., \\$V_{R_{sense}} \approx 8.55$ mV. Other current levels can be emulated in the same way.
    \item $V_{R_{sense}}$ will be up-converted using the non-inverting op-amp U1 to \\$V_{Nominal Input}=151\cdot8.55m\approx 1.27$ V. $V_{Nominal Input}$'s peak will be detected then that voltage will be fed into the non-inverting op-amp U2 to be scaled to a voltage between 0 and 5 VDC.
\end{itemize}

\begin{figure}[H]
    \centering
    \includegraphics[trim=0 5cm 0 4cm,scale=.4]{Figures/Current_overleaf1.pdf}
    \caption{Current transducer circuit}
    \label{fig:current_diag}
\end{figure}

The operational functionality of the voltage peak detection sub-circuit dominated by U3 and U4 of the current transducer circuit is discussed in section \ref{sec:design_vpd1} and will not be repeated in this section.
Op-amp characteristic voltages and surrounding components are considered and calculated as follows:
\begin{itemize}
 \item The common mode voltage, $V_{CM}$, for U3 and U4 which is the allowed input voltage at the positive and negative inputs of U3 and U4 are: $-5.3<V_{CM}<4$ [V] for supply voltages of \SI{+-5}{\volt DC} \cite{TLC2272}. Since $V_{Nominal Input}=1.3$ V at a maximum, the common mode voltages are satisfied at an ambient temperature, $T_A=$\SI{25}{\degreeCelsius} \cite{TLC2272}.\\The differential input voltage is $V_{diff}=$ \SI{+-16}{V_{peak}}, since there feedback on U3 and U4, the positive and negative inputs will be close to each other so that the differential input voltage specification is within bounds.
\item The feedback resistor is chosen as $R_1=20\ k\Omega$ for the same reason in Section \ref{sec:design_vpd1}, i.e.,that it needs to be big for efficient voltage feedback and chosen for small currents.
  \item Diodes $D_1$ and $D_3$ are chosen to be 1N4007 general diodes because they were the only provided. The forward resistance of $D_3$ is calculated as: $R_{diode}=\frac{V_T}{I_{diode}}=\frac{26m}{5\mu}\approx\SI{5.2}{\kilo\ohm}$, where $I_{diode}$ was chosen as the reverse peak current in
  \cite{1N4007}. We also know that to design for a fast transient of the tracking the peak we can choose\cite{voltage_peak_detector}: $R_{diode}\cdot C_1 \leq \frac{1}{10}T  \ \ \ \therefore C_2\approx384.6 \ nF \ \ \ \therefore choose \  C_2=100\ nF$ for better response time (although we are more concerned about choosing a smaller capacitance value for reducing the ripple than for increasing response time).
\end{itemize}

The following components were also chosen for the current transducer system as a whole:
\begin{itemize}
    \item $R_{sense}=30 \ m\Omega$ for better voltage increments to support an ADC resolution of 10 bits. This resistor was chosen over the $5 \ m\Omega$ supplied due to the resolution considerations for the ADC.
    \item Non-inverting gain op-amp U1 is chosen to amplify the voltage over  $R_{sense}$ for visible peak detection of the voltage peak detector. As we know, the voltage gain is calculated as: $A_v=1+\frac{R_3}{R_4}$. These resistors can be seen in Fig. \ref{fig:current_diag}. \\A voltage gain of 151 was chosen by setting $R_4=1 \ k\Omega$ and thus $R_3=150 \ k\Omega$ \\Non-inverting gain op-amp U2 is chosen to amplify the voltage at the output of the peak detector so that the output analogue voltage of the transducer would visibly be between 0 and 5 volt. As we know, the voltage gain is calculated as: $A_v=1+\frac{R_2}{R_5}$. These resistors can be seen in Fig. \ref{fig:current_diag}. \\A voltage gain of 3.2 was chosen by setting $R_5=1 \ k\Omega$ and thus $R_3=2.2\ k\Omega$.
\end{itemize}

In a functional summary, a 10 mA change decrease in $I_{R_{sense}}$ will make the output of the U1 decrease by some voltage $\delta V$. Then $V_{Nominal Input} \approx151\cdot\delta V$. This peak voltage is tracked by the voltage peak detector and finally fed into U2 Thus the output of U2, $V_{Nominal Output} \approx151\cdot\delta V\cdot3.2$. It must be noted that impact deviations on the input as a result of noise on rail and input voltages result in minor changes at the peak detector output of a few mV. These minor errors are amplified by the non-inverting amplifiers \cite{non-inverting} and cause a few mV changes on $V_{Nominal Output}$, but not out of project tolerances \cite{assignment_2}.

\section{Simulation} \label{sec:simulation_current}

 %ref appen here
The graphs in Fig. \ref{fig:nominal_current} (Appendix B) show the nominal input at the input of the voltage peak detector of the current transducer, a step change of 10 mA is included to make sure that the requirement mentioned in the assignment guidelines \cite{assignment_2} is met. The output clearly shows a response within 1 second after 0.5 seconds. The circuit performs as expected for system input currents ranging between 0 and 285 \si{\milli\ampere} and thus the system functions properly.


\section{Measurements} \label{sec:measurements_current}

The current peak transducer's analogue output at a mid-range load of \SI{1}{\kilo\ohm} is viewed on channel 2 of the oscilloscope is shown in Fig.\ref{subfig:current_sig}. When comparing this output to the analogue output in Fig. \ref{fig:nominal_current} (Appendix B) for $t<0.5s$, they are very close meaning the design is successful. The AC component shown in Fig. \ref{subfig:current_noise} does \textbf{not show} noise but rather the small-amplitude ripple of the DC output. This ripple with an a periodic amplitude of 30.4 mV helps track the peak voltage due to the time constant formed by $C_2$. Table \ref{cur_test_unit} summarizes data for when the system is stimulated with varying voltages, i.e, if it could respond to a 10 mA change in one second and also the test was performed to test the reliability of the op-amp for the required (emulated) current levels \cite{assignment_2}.


 \begin{figure}[H]
\centering
\begin{subfigure}{.5\textwidth}
  \centering
  \includegraphics[width=\linewidth,height=5cm]{Figures/Current_Signals.JPG}
    \caption{Integrated test input peak signal  (Channel 1) and current output analogue signal (Channel 2)}
    \label{subfig:current_sig}
\end{subfigure}%
\begin{subfigure}{.5\textwidth}
  \centering
  \includegraphics[width=\linewidth-0.5cm,height=5cm]{Figures/Current_Noise.JPG}
   \caption{Noise on current transducer's analogue output}
    \label{subfig:current_noise}
\end{subfigure}
\label{current_practical}
\caption{Current transducer practical measurements}
\end{figure}

\begin{table}[H]
\begin{tabular}{|c|c|c|c|c|c|}
\hline
\begin{tabular}[c]{@{}c@{}}Emulated Level\\ {[}\si{\milli\ampere_{peak}}{]}\end{tabular} & \begin{tabular}[c]{@{}c@{}}Signal \\ Generator \\ {[}\si{\milli\volt}{]}\end{tabular} & \begin{tabular}[c]{@{}c@{}}Signal \\ Generator\\ {[}\si{\milli\volt}{]}\end{tabular} & \begin{tabular}[c]{@{}c@{}}Analogue\\  Output \\ {[}\si{\volt DC}{]}\end{tabular} & \begin{tabular}[c]{@{}c@{}}Deduced\\  Input \\ {[}\si{\milli\ampere_{peak}}{]}\end{tabular} & \begin{tabular}[c]{@{}c@{}}Difference \\ {[}\si{\milli\ampere}{]}\end{tabular} \\ \hline
0      &0.00  &0.00 &0.25  &17.25   &17.25\\ \hline
50     &1.50  &1.26 &0.77  &53.12   &3.12\\ \hline
100    &3.00  &2.25 &1.28  &88.30   &11.7\\ \hline
101    &3.04  &2.29 &1.30  &89.68   &11.32\\ \hline
102    &3.07  &2.32 &1.31  &90.37   &11.63\\ \hline
200    &6.00  &5.69 &2.75  &189.1   &10.90\\ \hline
285    &8.60  & 8.26&3.99  &270.8   &14.2\\ \hline

\end{tabular}
\caption{Current transducer unit-level tests}
\label{cur_test_unit}
\end{table}

 Table \ref{cur_test_integrated} summarizes information aimed at actual system-testing, i.e., when the \SI{20}{\volt AC} was connected to the current transducer's s inputs. For table \ref{cur_test_integrated} \textbf{no curve-fit} was used.

\begin{table}[H]
\begin{tabular}{|c|c|c|c|c|c|c|c|}
\hline
\begin{tabular}[c]{@{}c@{}}Measure-\\ ment\\ \\ Load\end{tabular} & \begin{tabular}[c]{@{}c@{}}Load\\  R1 \\ {[}\si{\ohm}{]}\end{tabular} & \begin{tabular}[c]{@{}c@{}}Load \\ R2\\ {[}\si{\ohm}{]}\end{tabular} & \begin{tabular}[c]{@{}c@{}}Meas-\\ ured \\ Input Vr \\ {[}\si{\milli\volt_{peak}}{]}\end{tabular} & \begin{tabular}[c]{@{}c@{}}Actual \\ Input\\ {[}\si{\milli\ampere_{peak}}{]}\end{tabular} & \begin{tabular}[c]{@{}c@{}}Analo-\\ gue\\ Output \\ {[}\si{\volt DC}{]}\end{tabular} & \begin{tabular}[c]{@{}c@{}}Deduced \\ Input\\  {[}\si{\milli\ampere_{peak}}{]}\end{tabular} & \begin{tabular}[c]{@{}c@{}}Differ-\\ ence \\ {[}\si{\milli\ampere}{]}\end{tabular} \\ \hline
None& open& none             &0.00  &0      &0.24 &0.02  &0.02\\ \hline
Full& 100& none              &9.50  &296    &4.53 &312.5 &16.50\\ \hline
Mid& 1k& none                &1.06  &30.40  &0.521&35.94 &5.54\\ \hline
Mid + $\delta$& 1k&24k       &2.16  &31.67  &0.522&36.00 &4.33\\ \hline
Mid + $2\delta$& 1k&12k      &2.52  &32.93  &0.555&38.29 &5.36\\ \hline
\end{tabular}
\caption{Current transducer integrated tests}
\label{cur_test_integrated}
\end{table}
 
For the deduced columns in tables \ref{cur_test_unit} and \ref{cur_test_integrated}, the following formula (obtained from the transfer function) was used to approximate the input:$$I_{R_{sense}}=\frac{V_{Nominal Output}}{(1+\frac{R_3}{R_4})\cdot(1+\frac{R_2}{R_5})}\cdot \frac{1}{R_{sense}}$$ where $V_{Nominal Output}$ is the DC analogue output and $I_{R_{sense}}$ is the current through the sense resistor in Fig. \ref{fig:current_diag}. The formula contains non-inverting op-amp gains of U1 and U2 as well as $R_{sense}$.